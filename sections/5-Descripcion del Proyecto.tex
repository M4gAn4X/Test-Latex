\section{Descripción del proyecto}
\subsection{Metodología}
El desarrollo del presente sistema se llevará a cabo bajo un enfoque iterativo e incremental, aplicando las prácticas del marco de trabajo ágil Scrum, el cual permite adaptar los procesos a los cambios y priorizar el valor entregado al usuario final.
El proyecto se estructura en sprints cortos, donde cada iteración genera un incremento funcional del producto.
Durante estos ciclos, el equipo realiza las ceremonias establecidas por Scrum: \textit{Sprint Planning, Daily Scrum, Sprint Review y Sprint Retrospective}, garantizando la mejora continua y la transparencia del proceso.

La gestión del proyecto se desarrollará mediante la plataforma Azure DevOps, la cual facilita la planificación de tareas, el seguimiento de historias de usuario y la documentación de los entregables.
El diseño de interfaces se implementará utilizando Figma, mientras que la codificación se desarrollará en Node.js con el framework Express.js para el backend y React.js para el frontend.
Como gestor de base de datos se empleará MySQL, administrado mediante MySQL Workbench, lo que permitirá una estructura relacional robusta y segura para el manejo de la información.

Este enfoque ágil promueve la comunicación continua, la entrega temprana de valor y la retroalimentación constante.

\subsection{Actividades principales.}
El proyecto contempla un conjunto de actividades que permitirán cumplir con los objetivos definidos y garantizar el desarrollo del sistema conforme a los requerimientos identificados. Entre las actividades más relevantes se incluyen:
\begin{itemize}
  \item Levantamiento de requisitos y definición de historias de usuario, según los roles del sistema: Administrador, Tutor, Verificador y Tutorados.
  
  \item Diseño de prototipos de interfaz (wireframes y mockups) para representar el flujo de navegación y validar la experiencia del usuario.
  
  \item Modelado de la base de datos, definiendo entidades, relaciones y claves primarias que estructuren correctamente la información.
  
  \item Implementación incremental de funcionalidades prioritarias, integrando tanto la capa visual (frontend) como la lógica de negocio (backend).
  
  \item Pruebas funcionales y validación con el Product Owner, para asegurar el cumplimiento de los criterios de aceptación definidos..
\end{itemize}
Estas actividades se distribuyen progresivamente en los sprints del proyecto, de modo que cada iteración represente un avance funcional y verificable del producto final.

\subsection{Infraestructura y recursos.}
El proyecto contará con una infraestructura técnica basada en tecnologías modernas de desarrollo web.
El entorno de trabajo estará conformado por las siguientes herramientas y recursos:
    \subsubsection{Backend – Node.js:}  Node.js es un entorno de ejecución basado en JavaScript que permite construir servidores eficientes y escalables. Se elige por su rendimiento, amplia comunidad y compatibilidad con Express y MySQL.
    
    \subsubsection{Framework – Express.js:}Express.js es un framework minimalista para Node.js que simplifica la creación de APIs y manejo de rutas. Se usa por su flexibilidad y estructura clara para proyectos web.
    
    \subsubsection{Base de Datos – MySQL:}MySQL es un sistema gestor de bases de datos relacional ampliamente utilizado. Se elige por su estabilidad, soporte SQL estándar y fácil integración con Node.js.
    
    \subsubsection{Frontend – React.js:}React.js es una biblioteca de JavaScript para construir interfaces de usuario dinámicas mediante componentes. Se utiliza por su rendimiento, modularidad y gran ecosistema. 
    
    \subsubsection{Estilos – CSS:}CSS permite definir el diseño visual de las páginas web. Se usa por su control directo sobre colores, tamaños y disposición de los elementos.
    
    \subsubsection{Entorno de desarrollo – Vite:}Vite es una herramienta moderna para desarrollo frontend que ofrece recarga rápida y compilaciones optimizadas. Se selecciona por su velocidad y compatibilidad con React.
    
    \subsubsection{Modelado de Base de Datos – MySQL Workbench: }MySQL Workbench es una herramienta visual para diseñar, modelar y administrar bases de datos MySQL. Se usa por su facilidad para generar diagramas y scripts SQL automáticamente.


\begin{figure}[H]
    \centering
    % Marco personalizado con control total
    \begin{tcolorbox}[colframe=tablaxHeader, % color del borde
                      colback=white,          % color de fondo
                      boxrule=1.5pt,          % grosor del borde
                      arc=0mm,                % esquinas redondeadas (opcional)
                      left=0pt, right=0pt, top=0pt, bottom=0pt, % sin margen interno
                      width=0.8\linewidth,    % ancho total del marco
                      boxsep=2pt]             % separación entre borde e imagen
        \includegraphics[width=\linewidth]{images/Logos_Herramientas de desarrolo.png}
    \end{tcolorbox}
    \caption{Herramientas usadas para el desarrollo del sistema}
    \label{fig:placeholder}
\end{figure}

\subsection{Evaluación y monitoreo.}
El monitoreo del proyecto se realizará mediante mecanismos de control establecidos en el marco Scrum.
Se llevarán a cabo reuniones de seguimiento por sprint, donde se revisarán los avances, impedimentos y resultados alcanzados en comparación con los objetivos definidos en el Sprint Planning.
Los tableros Kanban de Azure DevOps permitirán visualizar el progreso de las tareas en tiempo real, clasificándolas según su estado (To Do, In Progress, Done).

Además, se realizarán Sprint Reviews para la validación del incremento con el Product Owner, y Sprint Retrospectives para identificar oportunidades de mejora en la gestión y el desarrollo.
Finalmente, se evaluará el cumplimiento del proyecto a través de indicadores de rendimiento, como el porcentaje de historias completadas, la satisfacción del cliente y la calidad del código entregado.

\noindent El horario de reunión del equipo para el desarrollo del proyecto se muestra a continuación:

\begin{itemize}
    \item Martes - Jueves: \newline 07:00 am- 09:00 am
    \item Sábado: \newline 04:00 pm - 06:00 pm
    \item Viernes: \newline 10:00 pm - 12:00 am
\end{itemize}

Las reuniones de equipo de desarrollarán por Google Meet, el link de la sala virtual se muestra a continuación: \url{https://meet.google.com/zyw-spgy-ihd}