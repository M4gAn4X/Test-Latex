\section{Descripción de la problemática}
Se propone la implementación de un sistema de gestión de tutorías para la Escuela Profesional de Ingeniería Informática y de Sistemas. 
El sistema de información tiene cuatro roles importantes: el de administrador del sistema, el tutor, el verificador y el tutorado.

\subsection{Rol del administrador}
\begin{itemize}
  \item Encargado de crear el cronograma de tutorías para cada semestre.
  \item Asignar a los tutorados a un tutor.
  \item Cambiar tutorados entre tutores.
  \item Cambiar de tutor a un grupo de estudiantes.
  \item Puede realizar el cronograma de tutorías por semestre, indicando las fechas de realización.
  \item Puede imprimir relación de tutorados con su tutor y el ambiente en el que se realizará la tutoría.
  \item Visualizar los reportes históricos de las tutorías de un estudiante.
  \item Las tutorías programadas deben contener tres aspectos importantes:
  \begin{itemize}
    \item \textbf{Académico:} se debe poder jalar del centro de cómputo la ficha de seguimiento del alumno o subir el archivo digital.
    \item \textbf{Personal:} incluye las actividades extracurriculares y aspectos psicológicos del alumno.
    \item \textbf{Profesional:} muestra el desarrollo profesional dentro de la escuela (cursos, capacitaciones, trabajos, etc.), con referencias y observaciones. También se debe considerar si el estudiante es derivado para un mejor acompañamiento psicológico.
  \end{itemize}
  \item Se deben mantener los estándares de seguridad, ya que se almacenará información muy sensible de los estudiantes de la escuela.
  \item Puede visualizar las tutorías realizadas en los semestres anteriores.
\end{itemize}

\subsection{Rol del tutor}
\begin{itemize}
  \item Ingresa los datos a los diferentes tipos de tutoría: académica, personal y profesional.
  \item Puede imprimir constancia al tutorado de haber pasado la tutoría.
  \item Puede modificar las tutorías hasta el cierre del cronograma del semestre.
  \item Puede imprimir la relación de estudiantes que pasaron tutoría.
\end{itemize}

\subsection{Rol del verificador}
\begin{itemize}
  \item Puede visualizar la lista completa de estudiantes que pasaron tutoría en una fecha determinada.
  \item Puede realizar consultas de tutorías de los diferentes semestres y tipos de tutoría.
  \item Puede hacer seguimiento de tutorías de un estudiante.
  \item Puede realizar seguimiento de las tutorías asignadas a un tutor en distintos semestres.
\end{itemize}

\subsection{Rol del tutorado}
\begin{itemize}
  \item Puede recibir notificaciones automáticas con la fecha, hora y lugar de las tutorías programadas, para garantizar su asistencia oportuna.
  \item Puede seleccionar un horario disponible entre las opciones propuestas por su tutor.
  \item Puede visualizar el historial de tutorías realizadas y descargar las constancias correspondientes a cada sesión.
  \item Puede consultar las observaciones y recomendaciones registradas por el tutor, con el fin de conocer su progreso y aspectos a mejorar.
\end{itemize}

Se deben mantener los mecanismos de seguridad para que solo los interesados puedan visualizar los datos ingresados en el sistema de información.