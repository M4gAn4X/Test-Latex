\section{Implementación del Proyecto}

\subsection{Product Backlog}
El Product Backlog constituye la lista priorizada de requisitos funcionales y técnicos que guían el desarrollo incremental del sistema. Cada elemento se expresa mediante una épica, desglosada en historias de usuario y tareas que permiten alcanzar incrementos funcionales del producto.
El backlog se inspecciona y adapta de forma iterativa a lo largo de todo el proyecto, manteniendo trazabilidad entre épicas, historias y entregables por sprint.

\subsubsection{Sprint 1 – Planificación y Diseño Visual del Proyecto}
El Sprint 1, correspondiente al periodo del 09 al 17 de octubre de 2025, tuvo como propósito establecer los cimientos organizativos y visuales del Sistema de Información para Tutorías.
En esta etapa se definieron los objetivos del sistema, se configuró el entorno de trabajo colaborativo y se produjeron los primeros diseños de interfaz. Las épicas incluidas en este sprint fueron \textbf{E1. Planificación del Proyecto} y \textbf{E2. Diseño Visual y Estructura del Proyecto}.

%% Insertamos el Product Backlog del Sprint 1
\begin{table}[H]
\centering
\begin{tabular}{|m{2cm}| m{5cm}| m{7.5cm}|}
\hline
% ================= CABECERA =================
\textcolor{white}{\cellcolor{tablaxHeader}\textbf{Épica}} &
\textcolor{white}{\cellcolor{tablaxHeader}\textbf{Historia de Usuario}} &
\textcolor{white}{\cellcolor{tablaxHeader}\textbf{Tareas}}\\
\hline

% ================= E1 =================
\multirow{3}{*}{\makecell[cc]{\\ \\ \\ \\ E1:\\Planificación\\del Proyecto}} 
    & \cellcolor{tablaxStripe}HU1: Como equipo de desarrollo, queremos definir las herramientas tecnológicas que se utilizarán (lenguajes, frameworks, gestores de BD y entornos de diseño) para garantizar uniformidad y compatibilidad entre componentes.
    & \cellcolor{tablaxStripe}T1 – Seleccionar herramientas de desarrollo backend y frontend.\newline
      T2 – Seleccionar herramientas de diseño y prototipado.\newline
      T3 – Determinar gestor de base de datos y herramientas de modelado.\newline
      T4 – Documentar todas las herramientas seleccionadas en un informe formal.\\

    & HU2: Como Product Owner, quiero definir el alcance y objetivos del sistema para que el equipo tenga una visión común del producto y se oriente correctamente durante todo el proyecto.
    & T5 – Redactar la presentación, ámbito, problemática, objetivos y descripción del proyecto.\newline
      T6 – Revisar y validar el documento de presentación.\newline
      T7 – Elaborar el documento de planificación de Sprints y Product Backlog.\newline
      T8 – Revisar y aprobar los documentos del Sprint 1.\\

    & \cellcolor{tablaxStripe}HU3: Como Scrum Master, quiero establecer la organización de trabajo en Azure DevOps para coordinar las tareas, visualizar el progreso y mantener una comunicación efectiva del equipo.
    & \cellcolor{tablaxStripe}T9 – Crear el proyecto principal en Azure DevOps.\newline
      T10 – Configurar el tablero de trabajo (Boards).\newline
      T11 – Registrar las épicas del proyecto.\newline
      T12 – Verificar permisos y sincronización del equipo.\\
\hline

% ================= E2 =================
\multirow{2}{*}{\makecell[cc]{E2:\\Diseño visual\\y estructura\\del Proyecto}}
    & HU4: Como desarrollador, quiero diseñar un modelo de base de datos inicial para asegurar que el sistema gestione correctamente la información de alumnos, tutores y tutorías.
    & T13 – Diseñar el esquema base de la base de datos.\newline
      T14 – Definir relaciones entre las tablas.\newline
      T15 – Generar el script SQL del modelo de datos.\newline
      T16 – Exportar y documentar el diagrama entidad–relación.\\

    & \cellcolor{tablaxStripe}HU5: Como diseñador UX/UI, quiero crear los wireframes del sistema para visualizar el flujo principal de pantallas antes de pasar al desarrollo.
    & \cellcolor{tablaxStripe}T17 – Crear la cuenta en Figma.\newline
      T18 – Crear el proyecto de diseño en Figma.\newline
      T19 – Diseñar la pantalla de inicio de sesión.\newline
      T20 – Diseñar el panel principal del sistema.\newline
      T21 – Diseñar el formulario de registro de tutorías.\newline
      T22 – Diseñar la pantalla de listado de tutorías.\newline
      T23 – Revisar y validar los wireframes creados.\\
\hline
\end{tabular}

\caption{Product Backlog detallado correspondiente al Sprint 1}
\label{tab:planificacion-proyecto}
\end{table}

\subsubsection{Sprint 2 – Login y Módulo del Administrador - Parte I}
El Sprint 2, correspondiente al periodo del 20 de octubre al 3 de noviembre de 2025, tendrá como propósito implementar el módulo de autenticación de usuarios y gestión inicial del administrador.
En esta etapa se desarrollarán las funcionalidades de inicio de sesión, creación del cronograma de tutorías, asignación de tutorados a tutores y configuración de los tipos de tutoría.
Las épicas incluidas serán \textbf{E3. Sistema de Login y Roles y E4. Funciones del Administrador – Parte I}.

%% ===============================================
%% Insertamos el Product Backlog del Sprint 2
\begin{table}[H]
\centering
\begin{tabular}{|m{2.25cm}| m{5cm}| m{7.5cm}|}
\hline
% ***************** CABECERA *********************
\textcolor{white}{\cellcolor{tablaxHeader}\textbf{Épica}} &
\textcolor{white}{\cellcolor{tablaxHeader}\textbf{Historia de Usuario}} &
\textcolor{white}{\cellcolor{tablaxHeader}\textbf{Tareas}}\\
\hline

% ================= E1 =================
\multirow{2}{*}{\makecell[cc]{\\  E3:\\ Sistema de\\ Login y \\ Roles}}
    & HU28: Como product owner necesito que los integrantes del equipo aprendan y manejen herramientas de desarrollo de software para la elaboración del proyecto, con el fin de trabajar de manera paralela y colaborativa.
    & T - Capacitación de Eduardo \newline
    T - Capacitación de Marco\newline
    T - Capacitación de Franshesco\newline
    T - Capacitación de Jhoel\newline
    T - Capacitación de Fidel\\
    
    & \cellcolor{tablaxStripe}HU6: Como usuario, quiero iniciar sesión con mi correo institucional y contraseña, para acceder únicamente al panel correspondiente a mi rol (Administrador, Tutor, Verificador o Tutorado).
    & \cellcolor{tablaxStripe}
    T – \newline
    T – (Por definir) \newline
    T – \newline
    T – \\

    & HU7: Como usuario, quiero cerrar sesión de forma segura para evitar accesos no autorizados a mi cuenta.
    & T – \newline
      T – (Por definir)\newline
      T – \newline
      T – \\
\hline

% ================= E2 =================
\multirow{4}{*}{\makecell[cc]{\\ \\ \\ \\ E4:\\ Funciones del\\ Administrador\\ – Parte I}}
    & \cellcolor{tablaxStripe}HU8: Como administrador, necesito crear/editar/eliminar el cronograma del semestre para organizar fechas y sesiones de tutoría.
    & \cellcolor{tablaxStripe}
    T – \newline
    T – (Por definir) \newline
    T – \newline
    T – \\

    & HU9: Como administrador, necesito asignar tutorados a tutores para equilibrar la carga y formalizar la atención.
    & 
    T – \newline
    T – \newline
    T – \newline
    T – (Por definir) \newline
    T – \newline
    T – \newline
    T – \\

    & \cellcolor{tablaxStripe}HU10: Como administrador, necesito reasignar un grupo de tutorados a otro tutor para reorganizar eficientemente sin hacerlo uno por uno.
    & \cellcolor{tablaxStripe}
    T – \newline
    T – \newline
    T – (Por definir) \newline
    T – \newline
    T – \\

    &HU11: Como administrador, necesito definir y parametrizar los tipos de tutoría (Académica, Personal, Profesional) para estandarizar los registros de los tutores.
    &
    T – \newline
    T – (Por definir) \newline
    T – \newline
    T – \newline
    T – \\
\hline
\end{tabular}

\caption{Product Backlog detallado correspondiente al Sprint 2}
\label{tab:planificacion-proyecto}
\end{table}
%% ==============================================

\subsubsection{Sprint 3 – Módulo del Administrador - Parte II y Tutor}
El Sprint 3, correspondiente al periodo del \textbf{4 al 17 de noviembre de 2025}, tendrá como propósito completar las funcionalidades del administrador y desarrollar el módulo del tutor.
Se abordarán las operaciones de reasignación individual de tutorados, consulta de historial, registro y modificación de tutorías, así como la generación de constancias y reportes.
Las épicas incluidas serán \textbf{E5. Funciones del Administrador – Parte II y E6. Funciones del Tutor}.

%% Insertamos el Product Backlog del Sprint 3
\begin{table}[H]
\centering
\begin{tabular}{|m{2.25cm}| m{5cm}| m{7.5cm}|}
\hline
% ***************** CABECERA *********************
\textcolor{white}{\cellcolor{tablaxHeader}\textbf{Épica}} &
\textcolor{white}{\cellcolor{tablaxHeader}\textbf{Historia de Usuario}} &
\textcolor{white}{\cellcolor{tablaxHeader}\textbf{Tareas}}\\
\hline

% ================= E1 =================
\multirow{3}{*}{\makecell[cc]{\\ \\ \\E5:\\ Funciones \\del \\ Administrador \\– Parte II}} 
    & \cellcolor{tablaxStripe}HU11: Como administrador, necesito cambiar tutorados entre tutores para balancear la carga académica y mantener la organización del cronograma.
    & \cellcolor{tablaxStripe}
    T – \newline
    T – (Por definir) \newline
    T – \newline
    T – \\

    & HU12: Como administrador, necesito imprimir la relación de tutorados por tutor y ambiente para tener un control documentado de las asignaciones.
    & T – \newline
      T – (Por definir)\newline
      T – \newline
      T – \\
      
    & \cellcolor{tablaxStripe} HU13: Como administrador, necesito consultar las tutorías realizadas en semestres anteriores para revisar el historial académico de los estudiantes.
    & \cellcolor{tablaxStripe}
    T – \newline
      T – (Por definir)\newline
      T – \newline
      T – \\
\hline

% ================= E2 =================
\multirow{4}{*}{\makecell[cc]{\\ \\ \\ \\ E6:\\ Funciones\\ del \\ Tutor}}
    & HU14: Como tutor, necesito registrar las tutorías académicas, personales y profesionales para llevar un seguimiento integral del progreso de mis estudiantes.
    &
    T – \newline
    T – (Por definir) \newline
    T – \newline
    T – \\

    & \cellcolor{tablaxStripe}HU15: Como tutor, necesito editar las tutorías registradas antes del cierre del cronograma para corregir errores o agregar información adicional.
    & \cellcolor{tablaxStripe}
    T – \newline
    T – \newline
    T – \newline
    T – (Por definir) \newline
    T – \newline
    T – \newline
    T – \\

    & HU16: Como tutor, necesito generar constancias PDF de tutorías para entregar al estudiante un comprobante oficial de su participación.
    & 
    T – \newline
    T – \newline
    T – (Por definir) \newline
    T – \newline
    T – \\

    &\cellcolor{tablaxStripe} HU17: Como tutor, necesito generar una lista general de los estudiantes atendidos para registrar oficialmente quiénes completaron sus tutorías. de los tutores.
    &\cellcolor{tablaxStripe}
    T – \newline
    T – (Por definir) \newline
    T – \newline
    T – \newline
    T – \\
\hline
\end{tabular}

\caption{Product Backlog detallado correspondiente al Sprint 3}
\label{tab:planificacion-proyecto}
\end{table}


%% ===============================================
\subsubsection{Sprint 4 – Módulo del Verificador y Tutorados}
El Sprint 4, correspondiente al periodo del \textbf{18 de noviembre al 1 de diciembre de 2025}, tendrá como propósito desarrollar los módulos del verificador y tutorado, permitiendo la interacción entre ambos roles.
Durante esta etapa se implementarán las funciones de consulta de tutorías, seguimiento académico, notificaciones automáticas y visualización de observaciones y constancias.
La épica incluida será \textbf{E7. Funciones del Verificador y Tutorados}.

%% Insertamos el Product Backlog del Sprint 4
\begin{table}[H]
\centering
\begin{tabular}{|m{2.25cm}| m{5cm}| m{7.5cm}|}
\hline
% ***************** CABECERA *********************
\textcolor{white}{\cellcolor{tablaxHeader}\textbf{Épica}} &
\textcolor{white}{\cellcolor{tablaxHeader}\textbf{Historia de Usuario}} &
\textcolor{white}{\cellcolor{tablaxHeader}\textbf{Tareas}}\\
\hline

% ================= E1 =================
\multirow{6}{*}{\makecell[cc]{\\ \\ \\ \\ \\ \\ \\ \\ \\E7: Funciones \\del\\ Verificador \\y Tutorados}} 
    & \cellcolor{tablaxStripe}HU18: Como verificador, necesito consultar las tutorías registradas por semestre para validar que se cumplan las sesiones académicas, personales y profesionales.
    & \cellcolor{tablaxStripe}
    T – \newline
    T – (Por definir) \newline
    T – \newline
    T – \\

    & HU19: Como verificador, necesito hacer seguimiento de las tutorías realizadas por un estudiante o un tutor para evaluar el cumplimiento y progreso de las actividades.
    & T – \newline
      T – (Por definir)\newline
      T – \newline
      T – \\
      
    & \cellcolor{tablaxStripe} HU20: Como tutorado, necesito recibir notificaciones automáticas con la fecha y hora de mis tutorías para asistir puntualmente y cumplir con las sesiones.
    & \cellcolor{tablaxStripe}
    T – \newline
      T – (Por definir)\newline
      T – \newline
      T – \\

    & HU21: Como tutorado, necesito seleccionar un horario disponible para mis tutorías para coordinar mi atención con el tutor asignado.
    & T – \newline
      T – (Por definir)\newline
      T – \newline
      T – \\

    & \cellcolor{tablaxStripe} HU22: Como tutorado, necesito visualizar mis tutorías pasadas y descargar mis constancias para llevar un registro personal de mis sesiones.
    & \cellcolor{tablaxStripe}
    T – \newline
      T – (Por definir)\newline
      T – \newline
      T – \\

    & HU23: Como tutorado, necesito ver las observaciones registradas por mis tutores para conocer mi progreso y aspectos a mejorar.
    & T – \newline
      T – (Por definir)\newline
      T – \newline
      T – \\
   
\hline
\end{tabular}

\caption{Product Backlog detallado correspondiente al Sprint 4}
\label{tab:planificacion-proyecto}
\end{table}


%% ===============================================
\subsubsection{Sprint 5 – Integración, Seguridad y Entrega Final}
El Sprint 5, correspondiente al periodo (por definir), tiene como objetivo principal el despliegue del proyecto, abarcando la integración de los distintos componentes desarrollados, la implementación de medidas de seguridad y la preparación para la entrega final al cliente. La épica incluída será \textbf{E8. Integración, Seguridad y Entrega Final}.

\begin{table}[H]
\centering
\begin{tabular}{|m{2.25cm}| m{5cm}| m{7.5cm}|}
\hline
% ***************** CABECERA *********************
\textcolor{white}{\cellcolor{tablaxHeader}\textbf{Épica}} &
\textcolor{white}{\cellcolor{tablaxHeader}\textbf{Historia de Usuario}} &
\textcolor{white}{\cellcolor{tablaxHeader}\textbf{Tareas}}\\
\hline

% ================= E1 =================
\multirow{6}{*}{\makecell[cc]{\\ \\ \\ \\ \\ \\ \\E8:Integración, \\Seguridad\\ y entrega \\final}} 
    & \cellcolor{tablaxStripe}HU24: Como usuario del sistema, necesito que el sistema aplique un control de acceso por rol para garantizar que cada usuario solo acceda a las funciones y datos que le corresponden.
    & \cellcolor{tablaxStripe}
    T – \newline
    T – (Por definir) \newline
    T – \newline
    T – \\

    & HU25: Como equipo de desarrollo, necesitamos integrar todos los módulos creados para entregar una versión completa y funcional del sistema de tutorías.
    & T – \newline
      T – (Por definir)\newline
      T – \newline
      T – \\
      
    & \cellcolor{tablaxStripe} HU26: Como Scrum Master, necesito realizar las pruebas finales y la retrospectiva del proyecto para garantizar la calidad del sistema y documentar las mejoras futuras.
    &\cellcolor{tablaxStripe}
      T – \newline
      T – (Por definir)\newline
      T – \newline
      T – \\

    & HU27: Como Product Owner, necesito compilar toda la documentación del sistema para entregar el producto final de manera formal y completa.
    & T – \newline
      T – (Por definir)\newline
      T – \newline
      T – \\
\hline
\end{tabular}

\caption{Product Backlog detallado correspondiente al Sprint 5}
\label{tab:planificacion-proyecto}
\end{table}
%% ===============================================
\begin{figure}[H]
    \centering
    \includegraphics[width=1\textwidth]{images/backlog_1.png}
    \caption{Produck Backlog realizado en Azure Dev Ops (1)}
    \label{fig:comparacion_tutorias}
\end{figure}
\begin{figure}[H]
    \centering
    \includegraphics[width=1\textwidth]{images/backlog_2.png}
    \caption{Produck Backlog realizado en Azure Dev Ops (2)}
    \label{fig:comparacion_tutorias}
\end{figure}

%% ======== SECCIÓN DE LOS SPRINT PLANNING =========
re\definecolor{micolor}{HTML}{4F94D9}

\section{Sprint Planning Document}
% Creamos la tabla inicial
\begin{table}[H]
\centering
\begin{tabular}{|m{2.5cm}|m{8.5cm}|} \hline

\cellcolor{micolor} \textcolor{white} {\bf{Proyecto:}} & Sistema de información para tutorías para la Escuela Profesional de Ingeniería Informática y de sistemas  \\ \hline

\cellcolor{micolor} \textcolor{white} {\bf{Sprint:}} & N° 01 \\ \hline

\cellcolor{micolor} \textcolor{white} {\bf{Duración:}} & 09 al 17 de octubre \\ \hline

\cellcolor{micolor} \textcolor{white} {\bf{Scrum \newline  Master:}} & Gallegos Silva Marco Abel \\ \hline

\cellcolor{micolor} \textcolor{white} {\bf{Product \newline Owner:}} & Magaña Osorio Jhoel Fabrizzio \\ \hline

\cellcolor{micolor} \textcolor{white} {\bf{Equipo de \newline desarrollo:}} & 
\begin{itemize}[noitemsep, topsep=2pt, itemsep=1pt, leftmargin=*, label=$\bullet$]
    \item Colque Quispe Fidel Enrique
    \item Santos Pillco Eduardo Jhosef
    \item Vasquez Mamani Raul Franshesco
\end{itemize} \\ \hline
   
\end{tabular}
\end{table}

\subsection{Objetivos del Sprint 1}
Establecer los cimientos organizativos y visuales del sistema “Sistema de Información para Tutorías”, definiendo el alcance del proyecto, configurando el entorno de trabajo colaborativo y elaborando los primeros prototipos visuales que orientarán las fases siguientes del desarrollo.

\subsection*{Épicas Incluidas}
\begin{itemize}[noitemsep, topsep=2pt, itemsep=1pt, leftmargin=*, label=$\bullet$]
    \item \bf{E1. Planificación del Proyecto}
    \item \bf{E2. Diseño Visual y Estructura del Proyecto}
\end{itemize}

\subsection*{Historias de usuario seleccionadas}
\begin{table}[H]
    \centering
    \begin{tabular}{|m{1.5cm}|m{3.5cm}|m{9cm}|} \hline
    
    \cellcolor{micolor} \textcolor{white}{\bf{Código}} & \cellcolor{micolor} \textcolor{white} {\bf{Nombre}} &
    \cellcolor{micolor} \textcolor{white} {\bf{Descripción}} \\ \hline 
    
    E1-HU1 & Definir herramientas de desarrollo y gestión & Como equipo de desarrollo, queremos definir las herramientas tecnológicas que se utilizarán; como lenguajes, frameworks, gestores de base de datos, entornos de diseño y despliegue, para garantizar uniformidad en el desarrollo y compatibilidad entre componentes. \\ \hline
    
    E1-HU2 & HU2: Definir alcance y objetivos del proyecto & Como Product Owner, quiero definir el alcance y objetivos del sistema para que el equipo tenga una visión común del producto que se va a desarrollar y se oriente correctamente durante todo el proyecto. \\ \hline 
    
    E1-HU3 & HU3: Organizar entorno de trabajo en Azure DevOps & Como Scrum Master, quiero establecer la organización de trabajo en la plataforma Azure DevOps para coordinar mejor las tareas, visualizar el progreso y mantener la comunicación efectiva del equipo. \\ \hline 
    
    E2-HU4 & HU4: Diseñar modelo de base de datos inicial & Como desarrollador, quiero tener una base de datos inicial bien estructurada para asegurar que el sistema gestione correctamente la información de alumnos, tutores y tutorías. \\ \hline 
    
    E2-HU5 & HU5: Diseñar wireframes principales del sistema & Como diseñador UX/UI, quiero crear los wireframes del sistema para visualizar el flujo principal de pantallas antes de pasar al desarrollo, facilitando la comprensión del diseño por parte del equipo y el cliente. \\ \hline
    
    \end{tabular}
    \caption{Historias de usuario seleccionadas para el Sprint 1}
    \label{tab:placeholder}
\end{table}

\subsection*{Desglose de tareas (Sprint Backlog)}

\begin{table}[h]
    \centering
    \begin{tabular}{|m{1.5cm}|m{9.5cm}|m{1.9cm}|}
    \hline

    \cellcolor{micolor} \textcolor{white}{\bf{Historia \newline Usuario}} & \cellcolor{micolor} \textcolor{white} {\bf{Tareas}} &
    \cellcolor{micolor} \textcolor{white} {\bf{Prioridad}} \\ \hline
    
    \multirow{4}{*}{E1-HU1} 
        & T1 – Seleccionar herramientas de desarrollo backend y frontend & \textcolor{white}{------}4\\ \cline{2-3}
        &T2 – Seleccionar herramientas de diseño y prototipado  &\textcolor{white}{------}4\\ \cline{2-3}
        &T3 – Determinar gestor de base de datos y herramientas de modelado &\textcolor{white}{------}4\\ \cline{2-3}
        & T4 – Documentar todas las herramientas seleccionadas en un informe formal&\textcolor{white}{------}3\\ \hline
         
    \multirow{4}{*}{E1-HU2}&T5 – Redactar la presentación, ámbito, problemática, objetivos y descripción del proyecto &\textcolor{white}{------}2\\ \cline{2-3}
         &T6 – Revisar y validar el documento de presentación &\textcolor{white}{------}1\\ \cline{2-3}
         &T7 – Elaborar el documento de planificación de Sprints y \newline Product Backlog &\textcolor{white}{------}2\\ \cline{2-3}
         &T8 – Revisar y aprobar los documentos del Sprint 1 &\textcolor{white}{------}1\\ \hline
         
    \multirow{4}{*}{E1-HU2}&T9 – Crear el proyecto principal en Azure DevOps&\textcolor{white}{------}3\\ \cline{2-3}
         &T10 – Configurar el tablero de trabajo (Boards) &\textcolor{white}{------}3\\ \cline{2-3}
         & T11 – Registrar las épicas del proyecto&\textcolor{white}{------}4\\ \cline{2-3}
         &T12 – Verificar permisos y sincronización del equipo &\textcolor{white}{------}2\\ \hline
         
    \multirow{4}{*}{E2-HU3}&T13 – Diseñar el esquema base de la base de datos&\textcolor{white}{------}3\\ \cline{2-3}
         &T14 – Definir relaciones entre las tablas &\textcolor{white}{------}3\\ \cline{2-3}
         &T15 – Generar el script SQL del modelo de datos&\textcolor{white}{------}3\\ \cline{2-3}
         &T16 – Exportar y documentar el diagrama entidad-relación&\textcolor{white}{------}3\\ \hline
         
    \multirow{5}{*}{E2-HU4}&T17 – Crear la cuenta en Figma&\textcolor{white}{------}3\\ \cline{2-3}
         &T18 – Crear el proyecto de diseño en Figma&\textcolor{white}{------}3\\ \cline{2-3}
         &T19 – Diseñar la pantalla de inicio de sesión&\textcolor{white}{------}3\\ \cline{2-3}
         &T20 – Diseñar el panel principal del sistema&\textcolor{white}{------}3\\ \cline{2-3}
         &T21 – Diseñar el formulario de registro de tutorías &\textcolor{white}{------}3\\ \cline{2-3}
         &T22 – Diseñar la pantalla de listado de tutorías &\textcolor{white}{------}3\\ \cline{2-3}
         &T23 – Revisar y validar los wireframes creados&\textcolor{white}{------}2\\ \hline 
    \end{tabular}
    \caption{Desglose de tareas del Sprint 1 (Sprint Backlog)}
    \label{tab:placeholder}
\end{table}

\subsection*{Definición de Hecho (Definition of Done)}
Para que una tarea o Historia de Usuario se marque o se considere como Done, se deben considerar los sigientes aspectos:
\begin{itemize}[noitemsep, topsep=2pt, itemsep=1pt, leftmargin=*, label=$\bullet$]
    \item Todas las historias de usuario cumplen sus criterios de aceptación.
    \item Los documentos y diseños fueron revisados y aprobados por el Product Owner.
    \item El entorno de trabajo en Azure DevOps está operativo con sus épicas y tareas registradas.
\end{itemize}

\subsection*{Incremento Esperado}
Al finalizar este Sprint, se debe contar y entregar los siguientes incrementos:
\begin{itemize}[noitemsep, topsep=2pt, itemsep=1pt, leftmargin=*, label=$\bullet$]
    \item Documento de planificación y objetivos del sistema.
    \item Entorno de trabajo configurado en Azure DevOps.
    \item Modelo de base de datos inicial validado.
    \item Prototipos visuales del sistema (wireframes).
\end{itemize}
    
%% =================================================

\subsection{Prototipo de interfaces}
    \subsubsection{Pantalla de inicio de sesión}
    \begin{figure}[H]
        \centering
        \includegraphics[width=0.75\linewidth]{images/inicio_sesion.png}
        \caption{Prototipo de inicio de sesión}
        \label{fig:placeholder}
    \end{figure}
    
    \subsubsection{Panel Principal del sistema}
    \begin{figure}[H]
        \centering
        \includegraphics[width=1\linewidth]{images/panel_tutorado.png}
        \caption{Prototipo del panel del tutorado}
        \label{fig:placeholder}
    \end{figure}

    \begin{figure}[H]
        \centering
        \includegraphics[width=1\linewidth]{images/panel_tutor.png}
        \caption{Prototipo del panel del tutor}
        \label{fig:placeholder}
    \end{figure}

    \begin{figure}[H]
        \centering
        \includegraphics[width=1\linewidth]{images/panel_administrador.png}
        \caption{Prototipo del panel del administrador}
        \label{fig:placeholder}
    \end{figure}

    \begin{figure}[H]
        \centering
        \includegraphics[width=1\linewidth]{images/panel_coordinador.png}
        \caption{Prototipo del panel del coordinador}
        \label{fig:placeholder}
    \end{figure}
    %\subsubsection{Formulario de Registro de tutorías}
    
    %\subsubsection{Panel de listado de Tutorías}
    
    %\subsubsection{Justificación del diseño visual}
    
    \subsection{Modelo de base de datos}   
        \subsubsection{Modelo lógico de la base de datos}
        \begin{figure}[H]
            \centering
            \includegraphics[width=1\linewidth]{images/bd_tutorias.png}
            \caption{Diagrama lógico entre la base de datos}
            \label{fig:placeholder}
        \end{figure}
        
        \subsubsection{Descripción de tablas principales}
        \noindent El modelo de la base de datos está conformado por seis tablas principales:
        \begin{itemize}
            \item USUARIOS
            \item TUTORES
            \item TUTORANDOS
            \item CRONOGRAMA
            \item TUTORIAS
            \item VERIFICACIONES
        \end{itemize}
        \subsubsection{Descripción de las relaciones}

        \textbf{USUARIOS → TUTORES (1 : 1)}
        
        Cada tutor está vinculado a un usuario del sistema. Un usuario con rol 'Tutor' solo puede tener un perfil de tutor. Si se elimina el usuario, también se elimina el registro del tutor.
        
        \textbf{USUARIOS → TUTORANDOS (1 : 1)}
        
        Cada tutorando (alumno) es también un usuario del sistema. El registro del tutorando depende del usuario correspondiente.
        
        \textbf{TUTORES → TUTORANDOS (1 : N)}
        
        Un tutor puede tener asignados varios tutorandos, pero cada tutorando solo tiene un tutor responsable durante el semestre.
        
        \textbf{USUARIOS → CRONOGRAMA (1 : N)}
        
        Un usuario con rol 'Administrador' puede crear varios cronogramas. Cada cronograma pertenece a un único usuario creador.
        
        \textbf{TUTORES → TUTORIAS (1 : N)}
        
        Un tutor puede registrar muchas tutorías (sesiones), pero cada tutoría pertenece a un solo tutor.
        
        \textbf{TUTORANDOS → TUTORIAS (1 : N)}
        
        Cada tutorando puede participar en varias tutorías durante el semestre, pero cada tutoría pertenece
        a un único alumno.
        
        \textbf{CRONOGRAMA → TUTORIAS (1 : N)}
        
        Un cronograma agrupa todas las tutorías programadas en las fechas oficiales del semestre.
        
        \textbf{USUARIOS → VERIFICACIONES (1 : N)}
        
        Un usuario con rol 'Verificador' puede realizar múltiples verificaciones, pero cada verificación la
        ejecuta un solo usuario.
        
        \textbf{TUTORIAS → VERIFICACIONES (1 : 1)}
        
        Cada tutoría tiene una única verificación asociada, en la que se valida y aprueba la sesión
        registrada. Si se elimina la tutoría, también se elimina su verificación.